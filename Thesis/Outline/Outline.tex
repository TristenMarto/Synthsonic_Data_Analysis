\documentclass{article}
\usepackage[utf8]{inputenc}

\author{Tristen}
\title{Outline \\ Synthsonic: Fast, probabilistic modeling and synthesis of tabular data}
\begin{document}

\maketitle

\section{Introduction (3-4 pages)}
    \begin{itemize}
        \item Introduction to imbalanced learning
        \item challenges of imbalanced classification
        \item problem statement
        \item research gap and research question
    \end{itemize}

\section{Literature review(3-4 pages)}
    \begin{itemize}
        \item discuss relevant work on oversampling techniques
        \item describe other generative synthesizers
    \end{itemize}


\section{background (4-6 pages)}
    \begin{itemize}
        \item Theoretic description of oversampling techniques: Start with the basic SMOTE version and explain the theory behind the technique
        \item potential drawbacks SMOTE-based oversamplers, ie how SMOTE variants rely on linear combinations of samples in order to generate new samples
    \end{itemize}

\section{Theory (6-10 pages)}
extensive description of synthsonic, explaining the steps in the oversampling process.
    \begin{itemize}
        \item mathematical transformation of features
        \item copula theory
        \item combination to BN
        \item discriminative classifier
        \item sampling method
    \end{itemize}

\section{Methods (3-5 pages)}
    \begin{itemize}
        \item description of global experiment design: How is the experiment set up? Explain reasoning for classifier, cross-validation, method of oversampling and evaluation of final results. Pseudo code of experiment run
        \item information about used datasets: Show the used datasets and explain their characteristics
        \item choice of metrics for experiments: describe choice of metrics and describe how they are calculated
        \item choice of oversamplers and parameters: proportion is varied and best performance is selected. Other parameters are kept constant.
    \end{itemize}
    
\section{Results (7-9 pages)}
        \begin{itemize}
        \item mean scores and overall results: Show a table ranking the oversamplers per metric. Show where each oversampler performs best 
        \item deep dive per dataset category: show results when analyzing the performance based on categorical, numerical or mixed datasets. Describe if there are any key differences.
        \item Bar charts of performance 
        \item PR-curve results: PR Curves show interesting patterns on very imbalanced data 
    \end{itemize}
    
    
\section{discussion (5-8 pages)}
discussion of results by five elements:
    \begin{itemize}
        \item Interpretations: what do the results mean?
        \item Implications: why do the results matter?
        \item Limitations: what can’t the results tell us?
        \item Recommendations: what are the next steps and topics to study?
        \item Relation to literature: two important comparison papers: "Oversampling Tabular Data with Deep Generative Models: Is it worth the effort?" (Camino R., 2020) and "An empirical comparison and evaluation of minority oversampling techniques on a large number of imbalanced datasets" (Kovacs G., 2019)
    \end{itemize}

\section{conclusion/future work (5-6 pages)}
    \begin{itemize}
        \item describe situations where synthsonic was better, and if it weighs up to the drawbacks
        \item early conclusions:
        \begin{itemize}
            \item standard XGBoost already performs well on imbalanced datasets.
            \item Synthsonic has higher runtimes than other oversamplers, the extra time is not always worth it.
            \item Synthsonic scores higher on categorical datasets, likely due to the fact it uses a Bayesian network to model all features instead of categorical only.
            \item overall, the effect of oversampling in general seems to be minimal. There is always an improvement, but very small in absolute numbers.
        \end{itemize}
    \end{itemize}

\end{document}
