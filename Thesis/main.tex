%%%%%%%%%%%%%%%%%%%%%%%%%%%%%%%%%%%%%%%%%
% Masters/Doctoral Thesis 
% LaTeX Template
% Version 1.43 (17/5/14)
%
% This template has been downloaded from:
% http://www.LaTeXTemplates.com
%
% Original authors:
% Steven Gunn 
% http://users.ecs.soton.ac.uk/srg/softwaretools/document/templates/
% and
% Sunil Patel
% http://www.sunilpatel.co.uk/thesis-template/
%
% License:
% CC BY-NC-SA 3.0 (http://creativecommons.org/licenses/by-nc-sa/3.0/)
%
% Note:
% Make sure to edit document variables in the Thesis.cls file
%
%
% Modified and Adapted by Michael Lees 2020
%%%%%%%%%%%%%%%%%%%%%%%%%%%%%%%%%%%%%%%%%

%----------------------------------------------------------------------------------------
%	PACKAGES AND OTHER DOCUMENT CONFIGURATIONS
%----------------------------------------------------------------------------------------

\documentclass[11pt, oneside]{Thesis} % The default font size and one-sided printing (no margin offsets)

\graphicspath{{Pictures/}} % Specifies the directory where pictures are stored

\usepackage[square, numbers, comma, sort&compress]{natbib} % Use the natbib reference package - read up on this to edit the reference style; if you want text (e.g. Smith et al., 2012) for the in-text references (instead of numbers), remove 'numbers' 
\usepackage[ruled,vlined]{algorithm2e}
\usepackage{amsmath}
\usepackage{booktabs}
\usepackage{svg}
\usepackage{caption}
\usepackage{subcaption}

\DeclareMathOperator*{\argmin}{arg\,min}
\DeclareMathOperator*{\argmax}{arg\,max}
\hypersetup{urlcolor=blue, colorlinks=true} % Colors hyperlinks in blue - change to black if annoying
\title{\ttitle} % Defines the thesis title - don't touch this

\begin{document}

\frontmatter % Use roman page numbering style (i, ii, iii, iv...) for the pre-content pages
\setstretch{1.3} % Line spacing of 1.3

% Define the page headers using the FancyHdr package and set up for one-sided printing
\fancyhead{} % Clears all page headers and footers
\rhead{\thepage} % Sets the right side header to show the page number
\lhead{} % Clears the left side page header

\pagestyle{fancy} % Finally, use the "fancy" page style to implement the FancyHdr headers

\newcommand{\HRule}{\rule{\linewidth}{0.5mm}} % New command to make the lines in the title page

% PDF meta-data
\hypersetup{pdftitle={\ttitle}}
\hypersetup{pdfsubject=\subjectname}
\hypersetup{pdfauthor=\authornames}
\hypersetup{pdfkeywords=\keywordnames}

%----------------------------------------------------------------------------------------
%	TITLE PAGE
%----------------------------------------------------------------------------------------

\begin{titlepage}
\begin{center}

\textsc{\LARGE \univname}\\[1.5cm] % University name
\textsc{\Large Masters Thesis}\\[0.5cm] % Thesis type

\HRule \\[0.4cm] % Horizontal line
{\huge \bfseries \ttitle}\\[0.4cm] % Thesis title
\HRule \\[1.5cm] % Horizontal line
 
\begin{minipage}{0.4\textwidth}
\begin{flushleft} \large
\emph{Author:}\\
{\authornames} % Author name
\end{flushleft}
\end{minipage}
\begin{minipage}{0.4\textwidth}
\begin{flushright} \large
\emph{Supervisors:} \\
{\supname} % Supervisor name
\end{flushright}
\end{minipage}\\[3cm]

\large \textit{A thesis submitted in partial fulfilment of the requirements\\ for the degree of \degreename}\\[0.3cm] % University requirement text
\textit{in the}\\[0.4cm]
\groupname\\\deptname\\[2cm] % Research group name and department name
 
{\large \today}\\[2cm] % Date
\includegraphics[width=0.6\textwidth]{Figures/clslogo.png} % Include Computational Science Logo
 
\vfill
\end{center}

\end{titlepage}

%----------------------------------------------------------------------------------------
%	DECLARATION PAGE
%	Your institution may give you a different text to place here
%----------------------------------------------------------------------------------------

\Declaration{

\addtocontents{toc}{\vspace{1em}} % Add a gap in the Contents, for aesthetics

I, \authornames, declare that this thesis, entitled `\ttitle' and the work presented in it are my own. I confirm that:

\begin{itemize} 
\item[\tiny{$\blacksquare$}] This work was done wholly or mainly while in candidature for a research degree at the University of Amsterdam.
\item[\tiny{$\blacksquare$}] Where any part of this thesis has previously been submitted for a degree or any other qualification at this University or any other institution, this has been clearly stated.
\item[\tiny{$\blacksquare$}] Where I have consulted the published work of others, this is always clearly attributed.
\item[\tiny{$\blacksquare$}] Where I have quoted from the work of others, the source is always given. With the exception of such quotations, this thesis is entirely my own work.
\item[\tiny{$\blacksquare$}] I have acknowledged all main sources of help.
\item[\tiny{$\blacksquare$}] Where the thesis is based on work done by myself jointly with others, I have made clear exactly what was done by others and what I have contributed myself.
\end{itemize}


Signed: 

\includegraphics[width=3cm]{Thesis/Figures/Handtekening.png}

Date: \today}

\clearpage % Start a new page

%----------------------------------------------------------------------------------------
%	ABSTRACT PAGE
%----------------------------------------------------------------------------------------

\addtotoc{Abstract} % Add the "Abstract" page entry to the Contents

\abstract{\addtocontents{toc}{\vspace{1em}} % Add a gap in the Contents, for aesthetics

Synthetic data, as the name suggests, is data which is artificially created rather than being generated by actual events. It’s often generated by algorithms or by AI methods and can be used as test data for new products or for AI model training. 
The main benefit of synthetic data is that it can be generated to meet specific conditions that are not available in (real) data. In this thesis the aim is to improve imbalanced learning algorithms by synthesizing new data to aid model learning. Current methods are difficult to train and rely solely on machine learning techniques which makes it less intuitive.
This novel approach was used in the classification of 27 highly imbalanced datasets and compared to other oversampling techniques. Results showed that oversampling resulted in a better classifier score, but that these improvements were minor when compared to the base performance. 
}

\clearpage % Start a new page

%----------------------------------------------------------------------------------------
%	ACKNOWLEDGEMENTS
%----------------------------------------------------------------------------------------

\setstretch{1.3} % Reset the line-spacing to 1.3 for body text (if it has changed)

\acknowledgements{\addtocontents{toc}{\vspace{1em}} % Add a gap in the Contents, for aesthetics

Thank

}
\clearpage % Start a new page

%----------------------------------------------------------------------------------------
%	LIST OF CONTENTS/FIGURES/TABLES PAGES
%----------------------------------------------------------------------------------------

\pagestyle{fancy} % The page style headers have been "empty" all this time, now use the "fancy" headers as defined before to bring them back

\lhead{\emph{Contents}} % Set the left side page header to "Contents"
\tableofcontents % Write out the Table of Contents

\lhead{\emph{List of Figures}} % Set the left side page header to "List of Figures"
\listoffigures % Write out the List of Figures

\lhead{\emph{List of Tables}} % Set the left side page header to "List of Tables"
\listoftables % Write out the List of Tables

\lhead{\emph{List of Algorithms}} % Set the left side page header to "List of Algorithms"
\addtotoc{List of Algorithms}
\listofalgorithms % Write out the List of Tables

%----------------------------------------------------------------------------------------
%	ABBREVIATIONS
%----------------------------------------------------------------------------------------

\clearpage % Start a new page

\setstretch{1.5} % Set the line spacing to 1.5, this makes the following tables easier to read

\lhead{\emph{Abbreviations}} % Set the left side page header to "Abbreviations"
\listofsymbols{ll} % Include a list of Abbreviations (a table of two columns)
{
\textbf{CSL} & \textbf{C}omputational \textbf{S}ceince \textbf{L}ab\\
\textbf{UvA} & \textbf{U}niversitiet \textbf{v}an \textbf{A}msterdam\\
\textbf{SMOTE} & \textbf{S}ynthetic \textbf{M}inority \textbf{O}versampling \textbf{TE}chnique \\
\textbf{GAN} & \textbf{G}enerative \textbf{A}dversarial \textbf{N}etwork \\
\textbf{ADASYN} & \textbf{ADA}ptive \textbf{SYN}thetic Sampling Approach \\
\textbf{KNN} & \textbf{K}-\textbf{Nearest} \textbf{Neighbours} 
%\textbf{Acronym} & \textbf{W}hat (it) \textbf{S}tands \textbf{F}or \\
}

%----------------------------------------------------------------------------------------
%	PHYSICAL CONSTANTS/OTHER DEFINITIONS
%----------------------------------------------------------------------------------------

\clearpage % Start a new page

%\lhead{\emph{Physical Constants}} % Set the left side page header to "Physical Constants"

%\listofconstants{lrcl} % Include a list of Physical Constants (a four column table)
%{
%Speed of Light & $c$ & $=$ & $2.997\ 924\ 58\times10^{8}\ \mbox{ms}^{-\mbox{s}}$ (exact)\\
%% Constant Name & Symbol & = & Constant Value (with units) \\
%}

%----------------------------------------------------------------------------------------
%	SYMBOLS
%----------------------------------------------------------------------------------------

%\clearpage % Start a new page

%\lhead{\emph{Symbols}} % Set the left side page header to "Symbols"

%\listofnomenclature{lll} % Include a list of Symbols (a three column table)
%{
%$a$ & distance & m \\
%$P$ & power & W (Js$^{-1}$) \\
%% Symbol & Name & Unit \\

%& & \\ % Gap to separate the Roman symbols from the Greek
%
%$\omega$ & angular frequency & rads$^{-1}$ \\
%% Symbol & Name & Unit \\
%}

%----------------------------------------------------------------------------------------
%	DEDICATION
%----------------------------------------------------------------------------------------

\setstretch{1.3} % Return the line spacing back to 1.3

\pagestyle{empty} % Page style needs to be empty for this page

\addtocontents{toc}{\vspace{2em}} % Add a gap in the Contents, for aesthetics

%----------------------------------------------------------------------------------------
%	THESIS CONTENT - CHAPTERS
%----------------------------------------------------------------------------------------

\mainmatter % Begin numeric (1,2,3...) page numbering

\pagestyle{fancy} % Return the page headers back to the "fancy" style

% Include the chapters of the thesis as separate  tex files from the Chapters folder
% Change the file names if you prefer

%This structure provides a bare bones essentials of the thesis and some indicative length
%This structure may not fit your thesis perfectly, but be sure to include these components somehow.
%It is possible to split the chapters up (E.g., 2 methods chapter, Experiment and Results as two, etc.)

% The typical length may be between 43 - 64 pages. Do not worry if you go slightly larger or smaller than this. 
% But a thesis of 20 pages, or 100 pages may suggest you've been to brief or too verbose.
%NOTE - there is not strict limit/minimum.
\chapter{Introduction}
\lhead{\emph{Introduction}}
Prediction and classification are two common tasks in data science. In some cases, these can be solved by using regression or other analytical tools. Other times, it is common practice to use machine learning algorithms to recognize patterns that we might miss. In practice, this could mean a classifier which determines if a patient has a disease or not, or a finance-related case, whether a transaction is fraudulent or not. Its application is varied, but the main task at hand stays the same: to distinguish between positive or negative, fraud or no fraud, sick or not sick. These problems are called binary classification, because a distinction must be made between two classes. This topic has been studied for a long time and many methods exist for it. 

\section{Difficulties in imbalanced classification}
However, the situation is not always as simple. In many real-life situations, predicting one class is more important than the other, and the weight of prediction errors are more significant than the other. In the example of medical testing, not detecting a disease weighs heavier than detecting a disease which is not present. This situation is often accompanied by the fact that the training data is imbalanced. This means that the distribution between classes is not symmetric, where one class largely outnumbers the other. In most cases, the minority class is important and needs to be predicted, which means that classifiers have a relatively low amount of data to recognize a pattern between these points. This issue of imbalanced learning is relevant everywhere and is therefore studied a lot and knows varied approaches. Studies on classification algorithms have tried to optimize the decision making process, but this is not the only approach. Another method is to modify the data at hand, by either adding or removing data to create a better decision boundary for the classifier. Data augmentation to aid imbalanced learning is the focal point of this thesis, it more specifically studies methods to generate synthetic data, creating new data points which resemble their original counterparts. 

\section{Research effort}
Generating synthetic data can be done in various ways. Here, a new method is proposed which generates new data samples which are independent from the original data samples, while keeping the same underlying probability distribution. The hypothesis is that these independent data samples will prevent overfitting in imbalanced datasets, and lead to an increased classifying performance. The following research question is introduced:

\begin{quote}
    To what extend does a probabilistic model creating independent new data samples aid in imbalanced classification performance?
\end{quote}

Classification performance can be interpreted in multiple ways. In the scope of this thesis, it is defined as the ability to correctly classify minority samples without sacrificing correctly classifying the majority samples.

\section{Outline of the thesis}
In this thesis, the topic of imbalanced classification is central. It aims to study different methods of oversampling and its effect on classifier performance. The topic will be introduced by a general view of existing methods for imbalanced learning, followed by a more detailed view on several oversampling methods. This leads to the literature review that examines relevant articles on data sampling techniques, using undersampling and oversampling, and comparative studies. It explains the relevance of this thesis, as well as the reason for a new oversampling model. Chapter 4 introduces Synthsonic, a probabilistic model for the synthesis of tabular data, and a competitor for existing oversamplers. This novel method will be tested on imbalanced datasets, and its performance will be measured against other methods. The experiments and results are presented in chapters 5 and 6, followed by chapters summarizing the implications of this thesis and its suggestion for future study.
 %Set out your thesis and state your research question (5-8 pages)
\chapter{Background}
\lhead{Background}
The problem statement is defined in the previous chapter, along with the outline of this thesis. As described in the introduction, the focal point is oversampling imbalanced data to help in classification problems. The first step is to describe popular methods when dealing with imbalanced data, before moving to the proposed model.

\section{Dealing with imbalance}
There are three popular methods for imbalanced data that will be discussed here: cost-sensitive learning, data level preprocessing methods and algorithm-level approaches. Data level preprocessing includes oversampling and undersampling. These methods handle imbalanced data during three different phases in model training. Algorithm-level approaches focus on modifying the classifier learning procedure without altering the data itself~\cite{Fernandez2018LearningSets}. The process requires an understanding of the mechanics of the classifier which lead to a bias towards the majority class. Since there are many classifiers, each with their own alterations, this topic will be limited to decision trees.

The first method depends on selecting the right algorithm and understanding its limitations in order to improve it. Decision tree classifiers are simple and efficient, and offer an interpretable solution~\cite{Safavian1991AMethodology}. A complex decision is split into a union of several smaller decisions, leading to the shape of a tree. Due to their simplicity, it is also easy to alter them from the algorithm-level approach. Each decision in the tree is made using a split function. Changing this key component is therefore the most straightforward approach possible. 

Secondly, cost-sensitive learning is also an aspect of an algorithm-level approach, but it does not alter the core algorithm itself. Instead, it introduces a missclassification cost which penalizes mistakes during classifier training. 


\section{SMOTE basics}

Describe how SMOTE works: Point selection and linear combination algorithm.

\section{oversampling techniques}
This section will talk about oversampling in general and will take a closer look at specific oversamplers.

Dealing with imbalanced data can be done in different ways, there are undersampling techniques where you remove samples from the majority class. This has the drawback that important data may be removed. The main focus point of this thesis are oversampling techniques, where different methods are used in order to synthesize new samples for the minority class. Both methods have a similar end goal: make the dataset more balanced.

\chapter{Literature Review}
\lhead{Literature Review}
This section discusses earlier work on alternative oversampling methods compared to SMOTE. When SMOTE was introduced in 2002, it proved to be a successful way to increase sensitivity to the minority class~\cite{Chawla2002SMOTE:Technique}. During the years after its publication, many alternatives were introduced such as ADASYN~\cite{He2008ADASYN:Learning} and Borderline-SMOTE~\cite{Han2005Borderline-SMOTE:Learning}. However, all oversamplers still relied on generating linear combinations from the samples in the minority class, but changed either which points were selected or how these were combined. 


\section{Generative Adversarial Networks}
Then Goodfellow et al. introduced a novel way of oversampling using a Generative Adversarial Network~\cite{Goodfellow2014GenerativeNets}. This deep learning method consists of a generative model $G$ which captures the data's distribution, and a discriminative model $D$ which estimates if a generated sample comes from the training data instead of $G$\footnote{Detailed description of the architecture of GANs are omitted in this thesis as it is not relevant to Synthsonic}. This approach is now used for synthesizing video and images which are indistinguishable from their real counterparts. This led to GAN adaptations such as MedGAN~\cite{Armanious2018MedGAN:GANs}, aiding medical image analysis and image translation, and VeeGAN~\cite{Srivastava2017VEEGAN:Learning}. The latter put more emphasis on the training method which could replicate the distribution of the dataset more accurately. While progression was being made, there was still an issue. The reason that GANs work so well on images and video is that the data is in a continuous-domain, these models still struggled when dealing with mixed distributions or discrete samples according to Camino et. al in 2020~\cite{Camino2020OversamplingEffort}. Their study showed that the improvements per performance metric are often significant when ranking the methods, but that the improvement in absolute terms is minor, also compared to the extra effort. Despite this, there was still a clear potential for data synthesis. 

\section{precursor of Synthsonic}
An article which sparked the idea for Synthsonic described how to use a Conditional GAN for modelling tabular data~\cite{Xu2019ModelingGAN}. Xu et al. were one of the first to apply a GAN to tabular data which contained both continuous and discrete features. While other articles were mainly focused on the -validity- of replicated images and videos, this article shifted the focus on being able to replicate a dataset instead. In other words, replicating data became a more important metric than replicating an image. However, using generative models 

Bayesian Networks on the other hand are proven to be effective for discrete features, two well performing models which are mentioned in literature are PrivBN~\cite{Zhang2014PrivBayes} and CLBN~\cite{Chow1968ApproximatingTrees}. The latter even dating back to 1968, showing its use was proven a long time ago. 

\section{Relevant study}
In 2019, Kovacs' published an empirical study where 85 oversamplers were tested on 104 datasets~\cite{Kovacs2019AnDatasets}. To date, there does not seem to be a study with a similar scale of approach. In comparison, the original SMOTE article used 9 datasets and ADASYN used 6 datasets. In theory, selecting fewer datasets can lead to skewed results when the oversampler is more suited to the chosen datasets. The approach of Kovacs is therefore a baseline in ranking SMOTE-based oversampling techniques. One issue with the study that the 104 datasets are not all unique. Some datasets are originally a multi-class classification problem, but the study only performed binary classification. To do this, multi class datasets were split on a one-vs-all basis, combining other labels to create an artificial imbalanced dataset. This creates many small datasets, some with around 100 total samples, where there are very few samples in the minority class to use for model training. The study also focused on SMOTE-based oversamplers only and did not include any 

In summary, most SMOTE-based oversamplers rely on a combination of both oversampling and undersampling to achieve an optimal performance. So far there has not been a method which can deliver a significant improvement when used as an only solution. GANs and Bayesian Networks provide an alternative but have their own drawbacks as well; GANs perform very well on continuous data, but are difficult to train and struggle with discrete features. There is still an ongoing search for a method which can reliably synthesize tabular data that contains both continuous and discrete features. Synthsonic aims to be a state of the art solution which can satisfy these conditions for a one step oversampling solution. While the comparison between SMOTE-based oversamplers has already been made, there is no study yet which tested generative models in a similar fashion. Camino et al.'s critical view on oversampling compared four SMOTE variants with four GANs, but was limited to two datasets. Combining these points, this thesis studies the effectiveness of Synthsonic as a probabilistic oversampling technique and compares this to popular SMOTE techniques.

\chapter{Theory}
\lhead{Theory}
The literature discussed in the previous chapter provides a base for Synthsonic. While SMOTE is easy to use and GANs are very successful at generating images and sound, there is still room for a probabilistic model for synthesizing tabular data due to the limitations of regular SMOTE-based oversamplers and the difficulties of GANs.

\section{Synthsonic}
This section is dedicated to the theory behind Synthsonic. Contrary to other techniques such as SMOTE or other generative algorithms, Synthsonic uses reversible mathematical transformations, making use of as much information in the dataset as possible. This has the additional benefit that Synthsonic is easier to interpret than other GANs, where the synthesis of new samples happens through black box machine learning methods. Additionally, 


Synthsonic consists out of multiple reversible transformations of features. It can handle both numerical and categorical features, and will split a dataset $X$ into a numerical and categorical set, so that $X = (X_{num}, X_{cat})$. Modelling of the dataset happens in five steps:

\begin{enumerate}
    \item Numerical features are transformed into uniformly distributed ones, using reversible steps, from which the first-order correlations have been removed.
    \item Categorical and discretized numerical features are jointly modelled with a Bayesian network.
    \item Specific features can be selected for additional modelling.
    \item A discriminative learner is trained to model the residual discrepancies observed between the input data and the Bayesian network.
    \item The discriminative learner is calibrated to reweigh samples from the Bayesian network.
\end{enumerate}



\chapter{Methods}
\lhead{\emph{Methods}}

This section describes the methods used in this thesis. Several terms and metrics will be introduced to the reader which are important to understand the experiments and results. Firstly, an imbalanced dataset of size $N$ is divided into a minority $N_{min}$ and majority $N_{maj}$ part, based on the number of samples per class, where $N = N_{min} + N_{maj}$. In this thesis, the minority class may also be named as the positive class. The experiments investigate in what ways oversampling can improve a classifier score.  

\section{datasets}
In order to examine the performance of the oversamplers, 27 datasets are used from the imbalanced-learn package~\cite{Lemaitre2017Imbalanced-learn:Learning}. Almost all of these datasets are publicly available in the UCI Machine Learning Repository~\cite{Dua2017UCIRepository}, which offers a wide variety of datasets for various machine learning purposes. The Imbalanced-learn package provides a selected set of imbalanced datasets to systematically benchmark the performance of the oversamplers. The datasets are also binarized, resulting in two classes to predict. This means that datasets which are multiclass, such as 'wine\_quality', are combined in such a way that only two remain with an imbalanced amount of samples.

\begin{table}
    \centering
        \begin{tabular}{llrrrrr}
        \toprule
                {}& dataset &    size &  features &  numerical &  categorical & imbalance \\
                & & & & features & features & ratio \\
        \midrule
        1 &           ecoli &     336 &         7 &                   5 &                     2 &           8.6:1 \\
        2 & optical\_digits &    5620 &        64 &                   0 &                    64 &           9.1:1 \\
        3 &       satimage &    6435 &        36 &                  36 &                     0 &           9.3:1 \\
        4 &     pen\_digits &   10992 &        16 &                  16 &                     0 &           9.4:1 \\
        5 &        abalone &    4177 &        10 &                   7 &                     3 &           9.7:1 \\
        6 & sick\_euthyroid &    3163 &        42 &                   6 &                    36 &           9.8:1 \\
        7 &   spectrometer &     531 &        93 &                  93 &                     0 &          10.8:1 \\
        8 &    car\_eval\_34 &    1728 &        21 &                   0 &                    21 &          11.9:1 \\
        9 &         isolet &    7797 &       617 &                 610 &                     7 &          12.0:1 \\
        10 &       us\_crime &    1994 &       100 &                  99 &                     1 &          12.3:1 \\
        11 &     yeast\_ml8 &    2417 &       103 &                 103 &                     0 &          12.6:1 \\
        12 &          scene &    2407 &       294 &                 294 &                     0 &          12.6:1 \\
        13 &    libras\_move &     360 &        90 &                  90 &                     0 &          14.0:1 \\
        14 &   thyroid\_sick &    3772 &        52 &                   6 &                    46 &          15.3:1 \\
        15 &      coil\_2000 &    9822 &        85 &                   1 &                    84 &          15.8:1 \\
        16 &     arrhythmia &     452 &       278 &                 137 &                   141 &          17.1:1 \\
        17 & solar\_flare\_m0 &    1389 &        32 &                   0 &                    32 &          19.4:1 \\
        18 &            oil &     937 &        49 &                  39 &                    10 &          21.9:1 \\
        19 &     car\_eval\_4 &    1728 &        21 &                   0 &                    21 &          25.6:1 \\
        20 &   wine\_quality &    4898 &        11 &                  11 &                     0 &          25.8:1 \\
        21 &     letter\_img &   20000 &        16 &                   0 &                    16 &          26.2:1 \\
        22 &      yeast\_me2 &    1484 &         8 &                   6 &                     2 &          28.1:1 \\
        23 &        webpage &   34780 &       300 &                   0 &                   300 &          34.5:1 \\
        24 &    ozone\_level &    2536 &        72 &                  72 &                     0 &          33.7:1 \\
        25 &    mammography &   11183 &         6 &                   6 &                     0 &          42.0:1 \\
        26 &   protein\_homo &  145751 &        74 &                  74 &                     0 &         111.5:1 \\
        27 &     abalone\_19 &    4177 &        10 &                   7 &                     3 &         129.5:1 \\
        \bottomrule
        \end{tabular}
    \caption{Overview of imbalanced datasets involved in the study}
    \label{tab:df_info}
\end{table}

The datasets are shown in table~\ref{tab:df_info}, providing information about the size, number of numerical and categorical features and the imbalance ratio, which is calculated as

\begin{equation}
    \text{imbalance ratio} = \frac{N_{maj}}{N_{min}}
\end{equation}

The datasets are sorted from smallest to biggest ratio between $N_{maj}$ and $N_{min}$. It can be seen that the most balanced dataset has 8.6 samples in the majority class for every minority sample and ranges all the way to 129.5 samples to one. Other variations between datasets include the number of total samples, $N$, ranging from 336 samples in the smallest dataset 'ecoli' to 145.751 samples in the 'protein\_homo' dataset. Larger datasets benefit from the fact that they include more information about each class, whereas small datasets lack the information to discover regularities or patterns in the training data~\cite{Ali2013ClassificationReview}.

A final note is the difference between numerical and categorical features. This thesis labels continuous features as numerical and discrete, ordinal or binary features as categorical. Techniques such as SMOTE, which work on a k-nearest neighbours principle, are meant to be used with numerical features and expect samples in a continuous sample space, while other techniques such as Random Oversampling copy existing samples and therefore do not require specific feature types. These variations in the datasets help in analysis of oversamplers in different scenarios and can help understand where they perform best.

\section{oversamplers}
This thesis tested the performance of ten oversamplers over 27 datasets. Note that SMOTE-NC does not work for exclusively numerical or categorical data, and is only used on the datasets which contain both feature types. The reasoning for SMOTE-NC is that it is an adaption on regular SMOTE, which already works with numerical features. To include a baseline for comparison, the results without oversampling are included as well. Next, five oversamplers from the IMB-learn package were selected. Oversamplers in this package have been published for at least three years with over 200 citations, and have a proven usefulness. These are considered well-established oversamplers and are a good benchmark for Synthsonic. One additional oversampler is added from Kovacs' study~\cite{Kovacs2019AnDatasets}, called Polynomial Fit SMOTE. This oversampler ranked as number one in the study, with the highest results in most metrics. This completes the group of 10 oversamplers to be tested on the imbalanced datasets. 

\section{metrics}
There are already a large variety of widely accepted and standard measures for binary classification, many of these are not suitable when dealing with imbalanced data. Scores such as accuracy are based on the number of correctly classified samples, where the performance of the majority class can be over represented~\cite{Fernandez2018LearningSets}. A classifier which assigns the majority class to all instances will achieve a $99\%$ accuracy if the imbalance ratio is $99:1$. Another drawback is that it is assumed that every error is equally costly. In imbalanced classification it is often more important to correctly label the minority class and misclassifying these cases should be associated with a higher cost. A medical model which fails to predict if a patient has a disease is costlier than a false alarm. Therefore, the goal is to label more minority classes correctly while still maintaining a good score on the majority class. The performance of the classifier is measured as either predicting the correct class labels or associated probability that a sample belongs to a class. Correctly predicting the class labels can be visualized by a confusion matrix, representing the number of TP (true positive), TN (True negative), FP (false positive) and FN (false negative) samples, where $P = TP + FN$ and $N = TN + FP$. The following metrics are then derived from these five measures.

\textbf{G-mean:} the geometric mean of accuracy achieved on minority and majority instances
\begin{equation}
    G = \sqrt{\frac{TP}{P} \cdot \frac{TN}{N}}
\end{equation}

\textbf{Precision:} the ratio between the true positives and false positives. Intuitively it is the ability to not mislabel a sample in the positive class
\begin{equation}
    PR = \frac{TP}{TP + FP}
\end{equation}

\textbf{Recall:} the ratio between true positives and false negatives, or the ability to find all positive samples.
\begin{equation}
    RE = \frac{TP}{TP + FN}
\end{equation}

\textbf{$F_1$-score:} the harmonic mean of precision (PR) and recall (RE)
\begin{equation}
    F_1 = 2 \cdot \frac{PR \cdot RE}{PR + RE}
\end{equation}

\textbf{Balanced accuracy:} the average of recall obtained on each class, it is the accuracy score with class-balanced sample weights
\begin{equation}
    \frac{\frac{TP}{TP + FN} + \frac{TN}{TN + FP}}{2}
\end{equation}

\textbf{PR AUC:} the Area Under Precision-Recall Curve charactarizes the trade-off between precision and recall for different probability thresholds.

\section{Evaluation methods}
cross validation: Due to the small dataset sizes $N$ and lower amount of samples in $N_{min}$, using a single split between training and testing data can lead to a biased result depending on the split. To prevent overfitting, the oversamplers are evaluated over a stratified k-fold cross-validation using four folds to achieve reproducible and reliable results. This number is chosen based on the smallest dataset to ensure that there are enough samples of the positive class in the testing set, the folds are the same for each oversampler to ensure a fair comparison. First, the minority class in the training set is oversampled, with different sampling proportions, before fitting the classifier. After fitting and predicting, the balanced accuracy, geometric mean, F1 score and AUC of the PR curve are determined on an out of hold validation set. This process is repeated for each fold in the cross-validation and the average is taken for a final score.  %Details of your approach (10-15 pages)
%\input{Chapters/ExperimentsandResults} %Evaluation/testing of your hypothesis (10-15 pages)
%\lhead{Discussion}
The main goal of the thesis is to give a broad view of oversampling performance on a wide variety of highly imbalanced datasets. In doing so, it generalizes the performance leading to the loss of information from individual datasets. As the results have shown, performance between oversamplers is very similar with minor differences in scores. From this perspective, the thesis agrees with the article of Camino et al.~\cite{Camino2020OversamplingEffort}, stating that oversampling offers an improvement, but it is minor compared to the effort it requires. 

The scores achieved with XGBoost were often close to their oversampled versions. In cases were the scores were low, oversampling did not significantly improve it. This leads to the belief that the classifier itself is the key factor to performance, and adding additional samples does not provide the desired effect. The literature agrees with this finding, stating that XGBoost is robust against class imbalance~\cite{Camino2020OversamplingEffort}. It is not clear if this algorithm can benefit from the injection of synthetic samples tot the minority class.

This leads to the next point of oversampling with Synthsonic, as it seems that synthesizing new independent samples do not improve on other SMOTE-based oversamplers. While Synthsonic outperformed others on some datasets, it did not show a significant improvement averaged over all datasets. The hypothesis was that SMOTE-based oversamplers were at risk of overfitting in cases of low sample size. The independent samples generated by Synthsonic would counter this risk and therefore perform better. Based on the results from this thesis, Synthsonic performs best on small datasets with few features and low number of minority samples, with diminishing returns as datasets grow.

With the results showing that oversampling offers a minor improvement, it raises the question why SMOTE oversamplers are so popular in use. Kovacs et al.~\cite{Kovacs2019AnDatasets} listed 85 variants of SMOTE, showing how much time and effort has been put into this topic. This study shows two reasons for this: SMOTE oversamplers are quick, with runtimes under a second, and they are easy to use. Secondly, the results can be improved more by applying undersampling as well. Having said that, the findings of Kovacs et al. differ from the results presented here. The Geometric mean and f1 score for k-nearest neighbours in this thesis are significantly lower than in the study of Kovacs et al. A possible explanation for this lies in the datasets which were used. In Kovacs et al., datasets have a lower IR (meaning the classes are more balanced), datasets are smaller and contain fewer minority samples and have fewer features. These factors can all contribute to the performance difference. However, this study also finds that the relative performance between oversamplers is minor, and the original SMOTE algorithm is capable of good results. Additionally, using a robust classifier such as XGBoost offers a performance increase similar to a simple model with oversampling.  

A final point of discussion is that there were no evident characteristics in the datasets showing that oversampling would be beneficial. The datasets in this study varied in size, imbalance ratio and features, but no clear connection was found between these factors and oversampling. All oversamplers performed equally well, achieving the highest scores on some datasets but being outperformed on others. The only exception that can be made here is SMOTE-NC, which suffers from the limitation that it can't be used on exclusively numerical or exclusively categorical datasets.

In summary, this thesis offers a comparison between SMOTE-based techniques and Synthsonic, and shows that performance is often similar on highly imbalanced datasets. There are still questions on why and when oversampling is beneficial, but these might be answered in future studies.   %Evaluation/testing of your hypothesis (5-8 pages)
%\input{Chapters/ConclusionsandFutureWork} % (5-6 pages)
\clearpage

\chapter{Results}
\lhead{\emph{Results}}

This chapter discusses about the results from the experiment with different SMOTE variants. 

10 oversamplers, including Synthsonic, were used in classifying 27 imbalanced datasets shown earlier in table~\ref{tab:df_info}. Due to the large amount of results, it is not feasible to discuss all details of oversampling. For comparison, the average results are shown for all datasets, but also per dataset type. The datasets are separated based on the number of categorical and numerical features in that dataset.

\section{Ranking of oversamplers}
In order to create a fair ranking over all datasets, we have taken the best performance per metric per oversampler. Table~\ref{tab:ranking} overview of the performance over various datasets with different characteristics. The first assumption is that the base classifier without oversampling should perform the worst out of all on most metrics. The exception is precision, which is expected to be higher when there is a high amount of samples in the majority class. 

\begin{table}[ht]
\centering
    \resizebox{\textwidth}{!}{
    \begin{tabular}{llrlrlrlrlrlr}
    \toprule
    {rank} &        Oversampler &  balanced\_accuracy &        Oversampler &    G\_mean &        Oversampler &        f1 &        Oversampler &  precision &        Oversampler &    recall &        Oversampler &    pr\_auc \\
    \midrule
    1  &             ADASYN &           0.798285 &             ADASYN &  0.729639 &           SVMSMOTE &  0.626186 &     NoOversampling &   0.699163 &             ADASYN &  0.614653 &         Synthsonic &  0.649430 \\
    2  &              SMOTE &           0.797884 &              SMOTE &  0.729364 &             ADASYN &  0.623797 &  polynom\_fit\_SMOTE &   0.693186 &         Synthsonic &  0.614159 &           SVMSMOTE &  0.649420 \\
    3  &         synthsonic &           0.797308 &         Synthsonic &  0.726812 &              SMOTE &  0.621968 &  RandomOversampler &   0.686230 &              SMOTE &  0.614040 &             ADASYN &  0.649238 \\
    4  &           SVMSMOTE &           0.796564 &  RandomOversampler &  0.726309 &       Random\_SMOTE &  0.621696 &           SVMSMOTE &   0.681582 &  RandomOversampler &  0.613770 &  RandomOversampler &  0.648616 \\
    5  &  RandomOversampler &           0.796323 &       Random\_SMOTE &  0.725619 &    BorderlineSMOTE &  0.620457 &       Random\_SMOTE &   0.679765 &       Random\_SMOTE &  0.609360 &  polynom\_fit\_SMOTE &  0.647145 \\
    6  &       Random\_SMOTE &           0.795903 &           SVMSMOTE &  0.724931 &  RandomOversampler &  0.620094 &             ADASYN &   0.678240 &           SVMSMOTE &  0.608782 &              SMOTE &  0.647125 \\
    7  &    BorderlineSMOTE &           0.794202 &    BorderlineSMOTE &  0.724181 &         synthsonic &  0.614133 &    BorderlineSMOTE &   0.675135 &    BorderlineSMOTE &  0.605111 &       Random\_SMOTE &  0.645629 \\
    8  &  polynom\_fit\_SMOTE &           0.781970 &  polynom\_fit\_SMOTE &  0.697985 &  polynom\_fit\_SMOTE &  0.606298 &         synthsonic &   0.671432 &  polynom\_fit\_SMOTE &  0.576800 &    BorderlineSMOTE &  0.642862 \\
    9  &            SMOTENC &           0.768076 &            SMOTENC &  0.693449 &     NoOversampling &  0.580367 &              SMOTE &   0.670155 &            SMOTENC &  0.564887 &     NoOversampling &  0.641915 \\
    10 &     NoOversampling &           0.761715 &     NoOversampling &  0.653643 &            SMOTENC &  0.555265 &            SMOTENC &   0.582476 &     NoOversampling &  0.531739 &            SMOTENC &  0.571079 \\
    \bottomrule
    \end{tabular}}
\caption{A ranking of oversamplers based on their average score of all datasets}
\label{tab:ranking}
\end{table}

Next we can see from figure~\ref{fig:Total} that No oversampling scores the lowest on all metrics, in accordance with our predictions. 

\begin{figure}[htp]
\centering
    \includesvg[width=.45\textwidth]{../Plots/Total/total_balanced_accuracy.svg}\quad
    \includesvg[width=.45\textwidth]{../Plots/Total/total_G_mean.svg}
    \medskip
    \includesvg[width=.45\textwidth]{../Plots/Total/total_f1.svg} \quad
    \includesvg[width=.45\textwidth]{../Plots/Total/total_pr_auc.svg}
    \medskip
    \includesvg[width=.45\textwidth]{../Plots/Total/total_precision.svg}\quad
    \includesvg[width=.45\textwidth]{../Plots/Total/total_recall.svg}
\caption{Average metrics scores per oversampler on all datasets. All scores range from 0 (worst) to 1 (best).}
\label{fig:Total}
\end{figure}

\section{Numerical datasets}
The first subcategory of datasets are the ones with either only numerical features or a majority of them. This subset contains 17 datasets, the majority of the total 27. Regular SMOTE techniques are expected to perform better here as their techniques require continuous features to work. Figure~\ref{tab:num}. Again ADASYN scores well on balanced accuracy and Geometric mean 

\begin{table}[ht]
\centering
\resizebox{\textwidth}{!}{
    \begin{tabular}{l|rr|rr|rr|rr|rr|rr}
        \toprule
        {} &  balanced accuracy &  std &  G\_mean & std &     f1 & std &  precision &  std &  recall &  std &  pr\_auc &  std \\
        \midrule
        ADASYN            &              0.758 &                  0.037 &   0.681 &       0.059 &  0.551 &   0.071 &      0.619 &          0.076 &   0.539 &       0.072 &   0.573 &       0.062 \\
        SMOTE             &              0.758 &                  0.032 &   0.680 &       0.054 &  0.548 &   0.056 &      0.605 &          0.066 &   0.538 &       0.067 &   0.568 &       0.060 \\
        SVMSMOTE          &              0.757 &                  0.037 &   0.674 &       0.067 &  0.555 &   0.070 &      0.626 &          0.073 &   0.532 &       0.073 &   0.574 &       0.058 \\
        Random\_SMOTE      &              0.755 &                  0.030 &   0.675 &       0.051 &  0.547 &   0.050 &      0.619 &          0.093 &   0.532 &       0.060 &   0.564 &       0.057 \\
        BorderlineSMOTE   &              0.753 &                  0.036 &   0.674 &       0.059 &  0.548 &   0.065 &      0.619 &          0.075 &   0.527 &       0.071 &   0.567 &       0.064 \\
        RandomOversampler &              0.747 &                  0.037 &   0.649 &       0.057 &  0.542 &   0.063 &      0.647 &          0.087 &   0.512 &       0.074 &   0.573 &       0.052 \\
        Synthsonic        &              0.747 &                  0.037 &   0.655 &       0.063 &  0.544 &   0.059 &      0.642 &          0.087 &   0.510 &       0.076 &   0.577 &       0.054 \\
        polynom\_fit\_SMOTE &              0.738 &                  0.032 &   0.638 &       0.050 &  0.526 &   0.060 &      0.637 &          0.089 &   0.492 &       0.066 &   0.567 &       0.051 \\
        SMOTENC           &              0.732 &                  0.041 &   0.643 &       0.075 &  0.489 &   0.063 &      0.522 &          0.092 &   0.497 &       0.081 &   0.493 &       0.057 \\
        NoOversampling    &              0.711 &                  0.037 &   0.577 &       0.065 &  0.494 &   0.070 &      0.653 &          0.079 &   0.432 &       0.075 &   0.566 &       0.059 \\
    \bottomrule
    \end{tabular}}
\caption{Scores per oversampler on datasets containing exclusively or a majority of numerical features.}
\label{tab:num}
\end{table}

\section{Categorical datasets}
The second subcategory of datasets are the ones with either exclusively or a large majority of categorical datasets. This includes nine of the total 27 datasets. Results are shown in table, after selecting the highest score per dataset  The assumption was that methods such as Synthsonic or ADASYN, which don't rely on a nearest neighbour principle, to perform better on this subset.

\begin{table}[ht]
\centering
\resizebox{\textwidth}{!}{
    \begin{tabular}{l|rr|rr|rr|rr|rr|rr}
        \toprule
        {} &  balanced accuracy &  std &  G\_mean & std &     f1 & std &  precision &  std &  recall &  std &  pr\_auc &  std \\
        \midrule
        Synthsonic        &              0.882 &                  0.019 &   0.850 &       0.024 &  0.734 &   0.027 &      0.722 &          0.047 &   0.790 &       0.035 &   0.772 &       0.031 \\
        RandomOversampler &              0.879 &                  0.022 &   0.857 &       0.030 &  0.752 &   0.033 &      0.752 &          0.049 &   0.786 &       0.044 &   0.776 &       0.029 \\
        ADASYN            &              0.867 &                  0.020 &   0.812 &       0.036 &  0.748 &   0.041 &      0.779 &          0.051 &   0.744 &       0.038 &   0.780 &       0.028 \\
        SMOTE             &              0.866 &                  0.020 &   0.813 &       0.036 &  0.748 &   0.039 &      0.781 &          0.055 &   0.744 &       0.040 &   0.781 &       0.031 \\
        Random\_SMOTE      &              0.866 &                  0.020 &   0.811 &       0.035 &  0.749 &   0.038 &      0.784 &          0.055 &   0.741 &       0.037 &   0.784 &       0.028 \\
        SVMSMOTE          &              0.864 &                  0.019 &   0.812 &       0.035 &  0.747 &   0.036 &      0.777 &          0.055 &   0.739 &       0.037 &   0.777 &       0.033 \\
        BorderlineSMOTE   &              0.864 &                  0.020 &   0.809 &       0.036 &  0.743 &   0.035 &      0.771 &          0.049 &   0.737 &       0.040 &   0.772 &       0.035 \\
        polynom\_fit\_SMOTE &              0.856 &                  0.020 &   0.801 &       0.035 &  0.742 &   0.033 &      0.788 &          0.053 &   0.721 &       0.041 &   0.784 &       0.032 \\
        NoOversampling    &              0.847 &                  0.025 &   0.784 &       0.043 &  0.726 &   0.040 &      0.778 &          0.055 &   0.702 &       0.050 &   0.772 &       0.035 \\
        SMOTENC           &              0.832 &                  0.030 &   0.782 &       0.037 &  0.671 &   0.050 &      0.688 &          0.073 &   0.684 &       0.057 &   0.708 &       0.055 \\
    \bottomrule
    \end{tabular}}
\caption{Scores per oversampler on datasets containing exclusively or a majority of categorical features.}
\label{tab:categorical}
\end{table}

The first observation is that SMOTE-NC performs poorly on these datasets, even scoring the lowest on F1-score, PR AUC and precision. Note that SMOTE-NC does not work on exclusively categorical features, and therefore only was used on three instead of nine datasets. Despite this, we want to portray how oversamplers perform in a wide variety of situations and try to find an overall best oversampler. For this reason the results of SMOTE-NC are included.

Comparing these results to those of the total datasets, we can see that Synthsonic scores higher on balanced accuracy and Geometric mean when compared to the total results. This means that Synthsonic is able to discern true positives and true negatives better due to oversampling. This is also reflected in the highest score in recall on categorical datasets, but when comparing to the total results we see a higher drop in precision. 

Contrary to the total results, there is a smaller trade-off between precision and recall.

\section{Mixed datasets}


\begin{table}[ht]
\centering
\resizebox{\textwidth}{!}{
    \begin{tabular}{l|rr|rr|rr|rr|rr|rr}
        \toprule
        {} &  balanced accuracy &  std &  G\_mean & std &     f1 & std &  precision &  std &  recall &  std &  pr\_auc &  std \\
        \midrule
        SMOTE             &              0.771 &                  0.032 &   0.690 &       0.052 &  0.557 &   0.055 &      0.589 &          0.075 &   0.569 &       0.068 &   0.573 &       0.061 \\
        ADASYN            &              0.771 &                  0.036 &   0.690 &       0.061 &  0.561 &   0.072 &      0.606 &          0.091 &   0.567 &       0.070 &   0.579 &       0.065 \\
        SVMSMOTE          &              0.769 &                  0.034 &   0.689 &       0.056 &  0.560 &   0.064 &      0.593 &          0.078 &   0.562 &       0.067 &   0.568 &       0.064 \\
        SMOTENC           &              0.768 &                  0.037 &   0.693 &       0.062 &  0.555 &   0.058 &      0.582 &          0.085 &   0.565 &       0.072 &   0.571 &       0.056 \\
        Random\_SMOTE      &              0.766 &                  0.032 &   0.682 &       0.048 &  0.555 &   0.052 &      0.615 &          0.107 &   0.558 &       0.063 &   0.570 &       0.056 \\
        BorderlineSMOTE   &              0.766 &                  0.036 &   0.683 &       0.058 &  0.551 &   0.060 &      0.577 &          0.070 &   0.557 &       0.074 &   0.559 &       0.066 \\
        RandomOversampler &              0.764 &                  0.038 &   0.686 &       0.049 &  0.550 &   0.058 &      0.583 &          0.077 &   0.560 &       0.077 &   0.567 &       0.052 \\
        Synthsonic        &              0.760 &                  0.042 &   0.670 &       0.063 &  0.543 &   0.058 &      0.585 &          0.079 &   0.542 &       0.086 &   0.574 &       0.057 \\
        polynom\_fit\_SMOTE &              0.740 &                  0.035 &   0.636 &       0.048 &  0.532 &   0.057 &      0.609 &          0.084 &   0.498 &       0.071 &   0.572 &       0.061 \\
        NoOversampling    &              0.726 &                  0.042 &   0.611 &       0.063 &  0.510 &   0.069 &      0.604 &          0.086 &   0.467 &       0.086 &   0.562 &       0.069 \\
    \bottomrule
    \end{tabular}}
\caption{Scores per oversampler on datasets containing exclusively or a majority of categorical features.}
\label{tab:mixed}
\end{table}


\section{Runtime}
This section discusses runtimes

The average runtime over all 27 datasets is shown in table~\ref{tab:runtimes}. These results are achieved by oversampling each dataset using a sampling strategy of 1, which means both classes contain an equal amount of samples. From the average runtimes it can be seen that the SMOTE-based oversamplers all have a comparable average runtime. The SMOTE-based oversamplers all average under 1 second for oversampling, with the exception of SMOTENC and SVMSMOTE. These oversamplers reached an average runtime of 1.395 seconds and 2.725 seconds respectively. From table~\ref{tab:runtimes} it can be seen that Synthsonic differs significantly from its counterparts. The average runtime was 86 seconds

\begin{table}[]
    \centering
    \begin{tabular}{lrr}
    \toprule
    stdev oversampler & runtime (s) &  stdev \\
    \midrule
    RandomOversampler &    0.014 &        0.001 \\
    polynom\_fit\_SMOTE &    0.041 &        0.002 \\
    SMOTE             &    0.059 &        0.002 \\
    BorderlineSMOTE   &    0.297 &        0.005 \\
    ADASYN            &    0.397 &        0.006 \\
    Random\_SMOTE      &    0.419 &        0.015 \\
    SMOTENC           &    1.395 &        0.024 \\
    SVMSMOTE          &    2.735 &        0.057 \\
    synthsonic        &   86.641 &        0.646 \\
    \bottomrule
    \end{tabular}
    \caption{Runtime for 1:1 oversampling, averaged over all 27 datasets.}
    \label{tab:runtimes}
\end{table}



-- overall view with boxplot and table


-- relation with size 

-- relation with features

-- 

\begin{figure}[h!]
\centering
\begin{subfigure}[b]{0.7\textwidth}
    \centering
    \includesvg[width=\textwidth]{../Plots/Runtime/boxplot.svg}
    \caption{A boxplot of the runtimes per oversampler over all datasets, oversampled to a ratio of 1:1.}
    \label{fig:boxplot}
\end{subfigure}
\vfill
\begin{subfigure}[b]{0.7\textwidth}
    \centering
    \includesvg[width=\textwidth]{../Plots/Runtime/runtime_vs_features.svg}
    \caption{The runtime for 1:1 oversampling, sorted from smallest to largest dataset.}
    \label{fig:runvfeat}
\end{subfigure}
\vfill
\begin{subfigure}[b]{0.7\textwidth}
    \centering
    \includesvg[width=\textwidth]{../Plots/Runtime/runtime_vs_size.svg}
    \caption{he runtime for 1:1 oversampling, sorted from fewest to most features.}
    \label{fig:runvsize}
\end{subfigure}

\end{figure}


\chapter{Discussion}
\lhead{Discussion}
The main goal of the thesis is to give a broad view of oversampling performance on a wide variety of highly imbalanced datasets. In doing so, it generalizes the performance leading to the loss of information from individual datasets. As the results have shown, performance between oversamplers is very similar with minor differences in scores. From this perspective, the thesis agrees with the article of Camino et al.~\cite{Camino2020OversamplingEffort}, stating that oversampling offers an improvement, but it is minor compared to the effort it requires. 

The scores achieved with XGBoost were often close to their oversampled versions. In cases were the scores were low, oversampling did not significantly improve it. This leads to the belief that the classifier itself is the key factor to performance, and adding additional samples does not provide the desired effect. The literature agrees with this finding, stating that XGBoost is robust against class imbalance~\cite{Camino2020OversamplingEffort}. It is not clear if this algorithm can benefit from the injection of synthetic samples tot the minority class.

This leads to the next point of oversampling with Synthsonic, as it seems that synthesizing new independent samples do not improve on other SMOTE-based oversamplers. While Synthsonic outperformed others on some datasets, it did not show a significant improvement averaged over all datasets. The hypothesis was that SMOTE-based oversamplers were at risk of overfitting in cases of low sample size. The independent samples generated by Synthsonic would counter this risk and therefore perform better. Based on the results from this thesis, Synthsonic performs best on small datasets with few features and low number of minority samples, with diminishing returns as datasets grow.

With the results showing that oversampling offers a minor improvement, it raises the question why SMOTE oversamplers are so popular in use. Kovacs et al.~\cite{Kovacs2019AnDatasets} listed 85 variants of SMOTE, showing how much time and effort has been put into this topic. This study shows two reasons for this: SMOTE oversamplers are quick, with runtimes under a second, and they are easy to use. Secondly, the results can be improved more by applying undersampling as well. Having said that, the findings of Kovacs et al. differ from the results presented here. The Geometric mean and f1 score for k-nearest neighbours in this thesis are significantly lower than in the study of Kovacs et al. A possible explanation for this lies in the datasets which were used. In Kovacs et al., datasets have a lower IR (meaning the classes are more balanced), datasets are smaller and contain fewer minority samples and have fewer features. These factors can all contribute to the performance difference. However, this study also finds that the relative performance between oversamplers is minor, and the original SMOTE algorithm is capable of good results. Additionally, using a robust classifier such as XGBoost offers a performance increase similar to a simple model with oversampling.  

A final point of discussion is that there were no evident characteristics in the datasets showing that oversampling would be beneficial. The datasets in this study varied in size, imbalance ratio and features, but no clear connection was found between these factors and oversampling. All oversamplers performed equally well, achieving the highest scores on some datasets but being outperformed on others. The only exception that can be made here is SMOTE-NC, which suffers from the limitation that it can't be used on exclusively numerical or exclusively categorical datasets.

In summary, this thesis offers a comparison between SMOTE-based techniques and Synthsonic, and shows that performance is often similar on highly imbalanced datasets. There are still questions on why and when oversampling is beneficial, but these might be answered in future studies.  

\chapter{Conclusion}
\lhead{Conclusion}
This thesis introduced the reader to binary classification and highlighted issues that go along with it. The literature showed that oversampling with SMOTE was widely used, and produced many different variants since its introduction in 2002. Because all these techniques relied on the same $k$-neighbours principle, it was thought this could lead to overfitting if there were few cases available. This led to Synthsonic, which uses invertible transformations and a Bayesian Network to synthesize independent data samples. Because these points were independent of the original data, it could offer an improvement over the SMOTE-based competitors. 

The results show that in highly imbalanced datasets, oversampling boosts the performance of the classifier in absolute terms, but are marginal compared to the baseline performance. Synthsonic manages to outperform other oversamplers in categorical datasets, most likely due to the inclusion of the Bayesian Network. However, this also led to high runtimes in datasets with many features, making it less usable in these situations. These runtimes are high compared to its SMOTE-based competitors, as they average runtime of under a second. The hypothesis that independent samples would result in better classifier performance could not be confirmed in this study. Performance varies per dataset, and there were no indications of when oversampling would be beneficial. This concludes that using XGBoost offers some protection against imbalanced classes, making the effect of oversampling small. One thing to add is that oversamplers boost recall, meaning that it makes the classifier more sensitive to positive cases, but often at a penalty in precision. This is a balance that needs to be addressed based on the situation and the application goal, especially when a simpler classification model (i.e. K-nearest neighbours) is used. Results showed that balanced accuracy and geometric mean are improved significantly, but that this does not hold for f1 score and PR-AUC. The relation between precision and recall becomes unbalanced, where oversampling yields a high recall but low precision. This means that the classifier becomes too sensitive to the minority class and overfits to these points. The results show that using a classifier with protection against imbalanced classes, such as XGBoost, counters this effect and gives the best results.

Synthsonic does prove a better alternative than other generative models, and provides an interpretable model which can easily be tuned to the situation. The transformations are all invertible, and can be removed if the situation does not need them. However, Synthsonic does not provide similar performance when compared to SMOTE-based oversamplers. There are still areas that still require more research to get a better understanding of oversampling. For one, this study could be repeated for datasets which are more balanced than the 27 datasets used here. As more data is available, it also provides more data for Synthsonic to improve its model which could result in higher performance. Secondly, Synthsonic performance is reliant on the Bayesian Network it creates, there is still room for improvement in the modeling of this network. Improving this part of the model would also lead to synthetic samples which resemble the original data more closely and could improve the performance further. For now, oversampling still seems a useful tool in classification, but is not yet the one-step-solution for imbalanced data.


%----------------------------------------------------------------------------------------
%	THESIS CONTENT - APPENDICES
%----------------------------------------------------------------------------------------

%\addtocontents{toc}{\vspace{2em}} % Add a gap in the Contents, for aesthetics

%\appendix % Cue to tell LaTeX that the following 'chapters' are Appendices

% Include the appendices of the thesis as separate files from the Appendices folder
% Uncomment the lines as you write the Appendices

%\input{Appendices/AppendixA}
%\input{Appendices/AppendixB}
%\input{Appendices/AppendixC}

\addtocontents{toc}{\vspace{2em}} % Add a gap in the Contents, for aesthetics

\backmatter

%----------------------------------------------------------------------------------------
%	BIBLIOGRAPHY
%----------------------------------------------------------------------------------------

\label{References}

\lhead{\emph{References}} % Change the page header to say "Bibliography"

\bibliographystyle{unsrtnat} % Use the "unsrtnat" BibTeX style for formatting the Bibliography

\bibliography{References.bib} % The references (bibliography) information are stored in the file named "Bibliography.bib"

\end{document}  